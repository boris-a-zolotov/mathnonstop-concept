\documentclass[a4paper,12pt]{article}
\usepackage[margin=2.5cm,left=2.15cm,top=1.4cm]{geometry}
\usepackage{amsmath,amssymb,mathtools}
\usepackage{enumerate,enumitem,indentfirst}
\usepackage{xspace,sectsty,hyperref}
\usepackage[russian]{babel}

\usepackage{mathspec}

\setmainfont[
	Path = f/,
	BoldFont=lb.ttf,
	ItalicFont=li.ttf,
	BoldItalicFont=lbi.ttf
		]{l.ttf}
\setsansfont[
	Path = f/,
	BoldFont=pb.ttf,
	ItalicFont=pi.ttf,
	BoldItalicFont=pbi.ttf
		]{p.ttf}
		
\setmathfont(Digits)[Path = f/]{csr.ttf}
\setmathfont(Latin)[Path = f/]{csi.ttf}
\setmathfont(Greek)[Path = f/, Uppercase]{t.ttf}
\setmathfont(Greek)[Path = f/, Lowercase]{ti.ttf}

\setmonofont[Path = f/]{pmono.ttf}
 \newcommand{\bz}{\mathbb Z} \newcommand{\br}{\mathbb R}
\newcommand{\vfi}{φ}
\newcommand{\divsby}{\mathop{\rlap{.}\rlap{\raisebox{0.55ex}{.}}\raisebox{1.1ex}{.}}}
\renewcommand{\ll}{\left(} \newcommand{\rr}{\right)}
\newcommand{\lag}{\left\langle} \newcommand{\rag}{\right\rangle}
\newcommand{\legendre}[2]{\ensuremath{\left( \frac{#1}{#2} \right)}}
\allsectionsfont{\normalfont\sffamily\bfseries}
\parindent=0.5in \parskip=0.1in \linespread{1.11}

\newcommand{\mns}{«Математика НОН-СТОП»\xspace}

\begin{document}

\begin{center} {\LARGE\bf \ \\
	Руководство по работе и взаимодействию \\
	площадок олимпиады \\
	\mns \\}
\end{center}

\begin{flushright} {\it
	Санкт-Петербург,\ \ 2020 год}
\end{flushright}

\section{Структура олимпиады \mns}

Основной структурной единицей олимпиады является {\itshape площадка}. Площадка~— школа или организация, где проводится олимпиада, на основании стратегического соглашения с Фондом «Время Науки». За площадкой закреплён список участников, которые зарегистрировались на неё для написания олимпиады. Наличие площадок олимпиады в разных районах Санкт-Петербурга и разных регионах РФ (и сопредельных государств) и возможность лёгкого масштабирования олимпиады посредством площадок является отличительной особенностью олимпиады \mns.

Организационными единицами олимпиады \mns, консолидирующими все площадки и отвечающими за её проведение, являются \vspace{-4mm}

\begin{itemize}
\item Методическая комиссия — отвечает за составление заданий и своевременное создание сборника решений для жюри и широкой аудитории читателей, а также за подготовку критериев для оценивания работ.

\item Организационный комитет — включает в себя руководителей площадок и волонтёров, отвечающих за время проведения олимпиады, размещение участников, распределение материалов.

\item Жюри — отвечает за проверку работ, формирование результатов и распределение наград.
\end{itemize}

Организациями, под эгидой которых проходит олимпиада \mns, являются \vspace{-4mm}

\begin{itemize}
\item Фонд «Время Науки» — материально обеспечивает олимпиаду, осуществляет её продвижение и снабжение сотрудниками.

\item Академия постдипломного педагогического образования — распространяет информацию об олимпиаде по школам и осуществляет поиск новых площадок.
\end{itemize}

\section{Информационные ресурсы, связанные с олимпиадой}

Сайт\ \ \url{mathnonstop.ru}\ \ содержит основную информацию об олимпиаде, необходимую для её участников, их родителей и учителей. На нём могут быть найдены итоги олимпиад последних лет, список площадок олимпиады, время начала и продолжительность олимпиады, ссылки на разборы заданий.

Регистрационная система\ \ \url{rs.mathnonstop.ru}\ \ служит для регистрации участников на площадки с одной стороны и сопоставления записей участников с их результатами, работами, наградами, персональными данными — с другой.

Регистрационная система обеспечивает работу оргкомитета, жюри и взаимодействие между площадками посредством обмена материалами, их централизованного хранения и обработки, а также взаимного контроля проверяющих и руководства жюри.

Страницы\ \ \url{timeforscience.ru}\ \ и\ \ \url{vk.com/timeforscience}\ \ содержат основную, общую и текущую информацию об организаторе олимпиады \mns — Фонде «Время Науки».

\section{Функции площадки олимпиады \mns}

\begin{itemize}
	\item Организация олимпиады \mns для участников из своего города.
	\item Увеличение ёмкости олимпиады \mns с тем, чтобы для всех желающих поучаствовать в олимпиаде нашлось место, где они могут принять в ней участие.
	\item Обеспечение удобства участникам олимпиады: чтобы каждый мог принять участие в олимпиаде не на значительном удалении от места жительства.
	\item Приобщение новых регионов и городов к математическим олимпиадам, а конкретно к олимпиаде \mns — для распространения опыта проведения олимпиад и разработки возможности проводить их самостоятельно.
	\item Распространение и продвижение информации о фонде «Время Науки», его мероприятиях, а также о самом учреждении, являющемся площадкой олимпиады.
\end{itemize}

\section{Обязанности площадок олимпиады и их руководителей}

\begin{itemize}
	\item Посещать (очно или онлайн) все вебинары и совещания, проводимые жюри и оргкомитетом олимпиады.
	\item Выделять аудитории для проведения олимпиады. Своевременно сообщать организационному комитету об имеющихся организационных возможностях, о числе и размере кабинетов и т.\,п.
	\item Выделять и распределять волонтёров в соответствии с выделенными аудиториями: по два волонтёра на аудиторию и до пяти волонтёров на вход на площадку и коридоры, а также проверяющих олимпиадных работ (волонтёрам и проверяющим будет выдан соответствующий сертификат).
	\item Печатать условия задач, регистрационные листы и списки участников непосредственно перед олимпиадой.
	\item Обеспечивать навигацию участников по площадке перед олимпиадой и во время её.
\end{itemize}

\section{Возможности площадок олимпиады и их руководителей}

\begin{itemize}
	\item Руководитель площадки самостоятельно выбирает, какое количество участников может принять, и сообщает это количество оргкомитету.
	\item Произвольное количество мест на площадке может быть зарезервировано под её учеников.
	\item На площадке могут проводиться занятия по материалам проведённой там\linebreak олимпиады \mns, в том числе с использованием решений, написанных для жюри.
	\item При проведении олимпиады на площадке могут распространяться рекламные материалы, подготовленные площадкой.
\end{itemize}

\section{В день проведения олимпиады}

\begin{itemize}
	\item Время начала олимпиады в Московском часовом поясе установлено на 15:00 первой субботы марта. Если площадка находится не в Московском часовом поясе, то время начала олимпиады оговаривается с руководством оргомитета в индивидуальном порядке.
	\item Олимпиада проводится на площадке в соответствии с {\it инструкцией по проведению олимпиады.}
\end{itemize}

\section{Проверка работ}

\begin{itemize}
	\item Проверяющие для работ олимпиады назначаются на площадках олимпиады, также можно стать проверяющим независимо, безотносительно к конкретной площадке.
	\item Проверяющий назначается по результатам собеседования с руководителем площадки и, по возможности, с руководством жюри олимпиады \mns.
	\item Перед тем, как приступить к проверке работ, проверяющий должен ознакомиться с официальным разбором задач, составленным методической комиссией.
	\item Результаты проверки работ заносятся {\it исключительно} в таблицы, экспортированные из системы\ \ \url{rs.mathnonstop.ru}\ \ или полученные от руководства жюри в личном порядке.
	\item Проверенные работы остаются на руках у проверяющего до того, как проведено награждение олимпиады.
	\item Если участник не явился на олимпиаду и факт неявки подтверждается регистрационным листом, в таблицу ставится $-1$ итоговый балл.
	\item Спорные случаи, когда задача оценивается неполным баллом и при этом такая оценка не прописана в критериях проверки, должны быть прокомментированы в работе.
	\item По всем вопросам, касающимся решений олимпиады и сомнений относительно выставляемого балла, требуется обращаться к руководству жюри.
\end{itemize}

\section{Система оценки базовых работ}

Каждая задача базовых вариантов состоит из трёх пунктов — A, B и C. За полное решение пункта A даётся 3 балла, за пункт B --- 6 баллов, и за пункт C --- 9 баллов. В случае частичного решения ученику может быть дана часть баллов, в случае особенно оригинального решения жюри может добавить 1–2 балла сверх полного балла.

Традиционно по каждой задаче в зачёт идёт лишь один пункт — тот, за который участник набрал наибольшее число баллов. Этот факт закреплён в инструкции на первой странице условий олимпиады.

\section{Система оценки профильных работ}

Пусть профильная задача имеет $k$ пунктов, а профильный вариант олимпиады писали
$n$ участников. Участник номер $i$ за пункт номер $p$ получает веществен-\linebreak ное число
$r_{i,p} \in [0,1]$:
1 соответствует полному и правильному решению, 0 — полностью неверному или отсутствующему решению.

{\it Сложностью} пункта будем называть число, обратное сумме
качеств его решений всех участников, которые его решали:
	$$s_p = \frac{1}{3 + \sum\limits_{i=1}^n r_{i,p}}.$$

После вычисления всех сложностей $s_p$ сложности
{\it нормируются} — если есть слишком маленькие или слишком большие значения, они соответственно
увеличиваются или уменьшаются на усмотрение жюри, чтобы не допустить чрезмерного разброса баллов.

В частности, способом такой нормировки является прибавление числа 3 к знаменателю. С одной стороны,
это позволяет избегать чересчур больших значений сложности, с другой — считается, что три члена жюри
знают решения всех задач.

\def\Coef{\text{Coef}}
Теперь каждый пункт оценивается прямо пропорционально своей сложности так,
чтобы сумма оценок за все пункты одной задачи равнялась 120 баллам. Для пункта номер $p$ вычисляется
	$$\Coef_p = 120 \cdot \frac{s_p}{\sum\limits_{j=1}^{k} s_j}.$$

После этого количество баллов, полученных участником номер $i$, вычисляется по формуле
	$$b_i = \sum\limits_{p=1}^{k} \Coef_p \cdot r_{i,p}$$

Легко заметить, что если задача решена всеми, за неё участники получат значительно меньше, чем человек,
который единственный решил самый сложный пункт. Кроме того, такая система устойчива к непришедшим участникам.

\section{Взаимодействие при проверке работ}

В системе\ \ \url{rs.mathnonstop.ru}\ \ предусмотрено {\it хранилище} для загрузки фотографий, сканов проверенных работ и неимпортированных таблиц проверки. Оно играет ключевую роль при взаимодействии между проверяющими, потому что позволяет осуществлять взаимный контроль и контроль со стороны руководства жюри за качеством проверки, обмениваться опытом, а также убеждаться, что за одно и то же решение все члены жюри ставят один и тот же балл. Каждая работа, находящаяся в хранилище, привязана к учётной записи участника, его варианту и регистрационному номеру.

Каждый проверяющий обязан во время или по окончании проверки своей доли работ {\it загрузить} в хранилище не менее 10 сканов или наборов фотографий работ участников. Эти работы должны включать в себя: \vspace{-4mm}

\begin{itemize}
	\item Несколько случайных работ, равномерно распределённых по вариантам олимпиады,
	\item Работы со спорными оценками: содержащие неполные решения, не отражённые в критериях олимпиады,
	\item Работы по регистрационным номерам, указанным в индивидуальном порядке руководством жюри.
\end{itemize}

Также обязательны к загрузке все работы, на которые поступило заявление при процедуре аппеляции.

Каждый проверяющий обязан во время или по окончании проверки своей доли работ {\it отсмотреть} не менее 20–25 работ, сканов или наборов фотографий работ, находящихся в хранилище и принадлежащих к региону, отличному от его собственного. Эти работы могут быть как из тех вариантов, на которых специализируется данный проверяющий, так и из всех остальных вариантов.

Если проверяющий обнаруживает некорректно, по его мнению, выставленные баллы в проверенных работах в хранилище, он должен сообщить об этом руководству жюри и, по возможности, тому проверяющему, который выставил эти баллы.

Все таблицы проверки должны быть по окончании проверки загружены в хранилище или (при умении работать с файлами {\tt csv}) импортированы в систему\ \ \url{rs.mathnonstop.ru}.

\section{Награждение победителей и призёров олимпиады}

Минимальные баллы для получения дипломов одинаковы для участников на всех площадках олимпиады~\mns. Однако минимальные баллы для получения похвальных отзывов могут различаться от региона к региону и назначаться на площадках в соответствующем регионе. Также допускается награждение участников отдельными номинациями, установленными на местах в регионах.

Рекомендуется проведение очного награждения в каждом регионе, где проходит олимпиада~\mns. При проведении очного награждения в регионе призы и дипломы с живыми подписями и печатями высылаются туда из Санкт-Петербурга.

\end{document}