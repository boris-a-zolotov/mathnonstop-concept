\documentclass[a4paper,12pt]{article}
\usepackage[margin=2.5cm,left=2.15cm,top=1.4cm]{geometry}
\usepackage{amsmath,amssymb,mathtools}
\usepackage{enumerate,enumitem,indentfirst}
\usepackage{xspace,sectsty,hyperref}
\usepackage[russian]{babel}

\usepackage{mathspec}

\setmainfont[
	Path = f/,
	BoldFont=lb.ttf,
	ItalicFont=li.ttf,
	BoldItalicFont=lbi.ttf
		]{l.ttf}
\setsansfont[
	Path = f/,
	BoldFont=pb.ttf,
	ItalicFont=pi.ttf,
	BoldItalicFont=pbi.ttf
		]{p.ttf}
		
\setmathfont(Digits)[Path = f/]{csr.ttf}
\setmathfont(Latin)[Path = f/]{csi.ttf}
\setmathfont(Greek)[Path = f/, Uppercase]{t.ttf}
\setmathfont(Greek)[Path = f/, Lowercase]{ti.ttf}

\setmonofont[Path = f/]{pmono.ttf}
 \newcommand{\bz}{\mathbb Z} \newcommand{\br}{\mathbb R}
\newcommand{\vfi}{φ}
\newcommand{\divsby}{\mathop{\rlap{.}\rlap{\raisebox{0.55ex}{.}}\raisebox{1.1ex}{.}}}
\renewcommand{\ll}{\left(} \newcommand{\rr}{\right)}
\newcommand{\lag}{\left\langle} \newcommand{\rag}{\right\rangle}
\newcommand{\legendre}[2]{\ensuremath{\left( \frac{#1}{#2} \right)}}
\allsectionsfont{\normalfont\sffamily\bfseries}
\parindent=0.5in \parskip=0.1in \linespread{1.11}

\newcommand{\mns}{«Математика НОН-СТОП»\xspace}

\begin{document}

\section{Структура олимпиады \mns}

Основной структурной единицей олимпиады является {\itshape площадка}. Площадка~— школа или организация, где проводится олимпиада, на основании соглашения с Фондом «Время Науки» или устной договорённости. За площадкой закреплён список участников, которые зарегистрировались на неё для написания олимпиады. Наличие площадок олимпиады в разных районах Санкт-Петербурга и разных регионах РФ и возможность лёгкого масштабирования олимпиады посредством площадок является гордостью олимпиады \mns.

Организационными единицами олимпиады, консолидирующими все площадки и отвечающими за её проведение, являются \vspace{-4mm}

\begin{itemize}
\item Методическая комиссия — отвечает за составление и тестирование заданий, своевременное создание сборника решений для жюри и широкой аудитории читателей.

\item Организационный комитет — включает в себя руководителей площадок, а также волонтёров, отвечающих за время проведения олимпиады, размещение участников, распределение материалов.

\item Жюри — отвечает за проверку работ, формирование результатов и распределение наград.
\end{itemize}

Организациями, под эгидой которых проходит олимпиада \mns, являются \vspace{-4mm}

\begin{itemize}
\item Фонд «Время Науки» — материально обеспечивает олимпиаду, осуществляет её продвижение и снабжение сотрудниками.

\item Академия постдипломного педагогического образования — распространяет информацию об олимпиаде по школам и осуществляет поиск новых площадок.
\end{itemize}

\section{Информационные ресурсы, связанные с олимпиадой}

Сайт\ \ \url{mathnonstop.ru}\ \ содержит основную информацию об олимпиаде, необходимую для её участников, их родителей и учителей.

Регистрационная система\ \ \url{rs.mathnonstop.ru}\ \ служит для регистрации участников на площадки с одной стороны и сопоставления записей участников с их результатами, работами, наградами, персональными данными — с другой. В регистрационной системе происходит работа оргкомитета и жюри.

\end{document}