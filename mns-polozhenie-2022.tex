\documentclass[a4paper,12pt]{article}
\usepackage[margin=2.25cm,left=2.15cm,top=1.4cm]{geometry}
\usepackage{amsmath,amssymb,mathtools}
\usepackage{enumerate,enumitem,indentfirst}
\usepackage{xspace,hyperref}
\usepackage{makecell}
\usepackage[russian]{babel}

\usepackage{mathspec}

\setmainfont[
	Path = f-tnr/,
	BoldFont=tb.ttf,
	ItalicFont=ti.ttf,
	BoldItalicFont=tbi.ttf
		]{t.ttf}
		
\setmathfont(Digits)[Path = f-tnr/]{t.ttf}
\setmathfont(Latin)[Path = f-tnr/]{ti.ttf}
\setmathfont(Greek)[Path = f-tnr/, Uppercase]{g.ttf}
\setmathfont(Greek)[Path = f-tnr/, Lowercase]{gi.ttf}

\setmonofont[Path = f/]{pmono.ttf}
 \newcommand{\bz}{\mathbb Z} \newcommand{\br}{\mathbb R}
\newcommand{\vfi}{φ}
\newcommand{\divsby}{\mathop{\rlap{.}\rlap{\raisebox{0.55ex}{.}}\raisebox{1.1ex}{.}}}
\renewcommand{\ll}{\left(} \newcommand{\rr}{\right)}
\newcommand{\lag}{\left\langle} \newcommand{\rag}{\right\rangle}
\newcommand{\legendre}[2]{\ensuremath{\left( \frac{#1}{#2} \right)}}
\parindent=0.5in \parskip=0.1in \linespread{1.11}

\setlist{leftmargin=0.35in}

\renewcommand{\theenumi}{\thesection.\arabic{enumi}.}
\renewcommand{\labelenumi}{\theenumi}

\newcommand{\mns}{«Математика НОН-СТОП»\xspace}
\newcommand{\surl}[1]{{\small\url{#1}}}

\begin{document}

\thispagestyle{empty}

\begin{center} \large\bf \ \\
	Фонд поддержки научной и научно-технической деятельности \\
	молодых учёных «Время Науки» \bigskip

	Государственное бюджетное учреждение \\
	дополнительного профессионального образования \\
	«Академия постдипломного педагогического образования» \bigskip

\normalsize
\begin{tabular}{ccc}
	\makecell[l]{
		УТВЕРЖДЕНО \\
		Решением Директора Института \\
		Общего образования \\
		Государственного бюджетного учреждения \\
		дополнительного профессионального образования \\
		Санкт-Петербургской академии постдипломного \\
		педагогического образования		
	} & \hspace{0.05cm} &
	\makecell[r]{
		УТВЕРЖДЕНО \\
		Решением Совета Фонда \\
		поддержки научной \\
		и научно-технической \\
		деятельности молодых \\
		учёных «Время Науки»
	} \\ \ & \ \\
	\makecell[l]{\ \\
		Директор Института общего образования \\
		ГБУ ДПО «СПбАППО» \\ \\
		\underline{\hspace{2.5cm}}\ C. В. Алексеев
	} & &
	\makecell[r]{
		Президент Фонда \\
		«Время Науки» \\
		Председатель Орг. комитета \\
		Олимпиады \\
		\underline{\hspace{2.5cm}}\ И. А. Чистяков
	} \\ \ & \ \\
	\makecell[l]{«\underline{\hspace{1.25cm}}»\ 
		\underline{\hspace{1.75cm}}\ 
		2022 г.}
\end{tabular} \end{center}

\vspace{1cm}

\begin{center} {\LARGE\bf \ \\
	Открытая \\
	математическая олимпиада \\
	\mns \\ [1 cm]
	ПОЛОЖЕНИЕ}
\end{center}

\vfill

\begin{center} \bf
	Санкт-Петербург \\
	2022 год
\end{center}

\newpage


\begin{center} \Large
	{\bf ПОЛОЖЕНИЕ} \\
	об открытой олимпиаде \\
	\mns \\
\end{center}

\section{Общие положения}

\begin{enumerate}
	\item Санкт-Петербургская открытая городская олимпиада «Математика НОН-СТОП» (далее~— Олимпиада)~— это индивидуальное соревнование школьников в умении решать математические задачи повышенной сложности.
	\item Цель Олимпиады заключается в подготовке и привлечении российских школьников к исследовательской работе и регулярным занятиям в области математики. Олимпиада является важным звеном в формировании у учащихся элементов научного мышления.
	\item Олимпиада способствует: \begin{enumerate}
	   \item[–] Привлечению научно-педагогической общественности Санкт-Петербурга и России к работе с одаренной молодежью,
	   \item[–] Созданию комплекса научно-методических материалов для организации исследовательской деятельности школьников в области математики,
	   \item[–] Построению концепции непрерывного образования в сфере проектной и
исследовательской деятельности учащихся,
	   \item[–] Повышению мотивации учащихся к занятию математикой.
   \end{enumerate}
	\item Соучредителями Олимпиады являются Фонд поддержки научной и научно-технической деятельности молодых ученых «Время науки» и Государственное бюджетное учреждение дополнительного профессионального образования «Санкт-Петербургская академия постдипломного педагогического образования». Партнером выступает Частное образовательное учреждение общего и дополнительного образования «Лаборатория непрерывного математического образования».
	\item Общее руководство Олимпиадой осуществляет Фонд «Время науки». 
	\item Для достижения целей и задач Фонд «Время науки» проводит Олимпиаду для обучающихся 4–8 классов средних общеобразовательных учреждений, зарегистрировавшихся для участия в Олимпиаде.
	\item Олимпиада является очной и проводится в один этап.
	\item Участие в Олимпиаде бесплатно.
	\item Основной структурной единицей Олимпиады является площадка. Площадка~— школа или организация, где проводится Олимпиада, на основании стратегического соглашения с Фондом «Время науки».
	\item Список актуальных площадок доступен на сайте Олимпиады.
	\item Организационными единицами Олимпиады, отвечающими за ее проведение и консолидирующими все площадки, являются: \begin{enumerate}
	   \item[–] Организационный комитет Олимпиады: включает в себя руководителей площадок и волонтеров, отвечающих за подготовку площадки к проведению Олимпиады, навигацию по площадке, процесс проведения Олимпиады, размещение участников и их родителей, заполнение регистрационных листов, распределение материалов и справочных брошюр, формирование пакетов документов, соответствующих аудиториям.
	   \item[–] Методическая комиссия Олимпиады: в обязанности методической комисcии входит составление заданий и своевременное создание сборника решений для жюри и широкой аудитории читателей, а также подготовка критериев для оценивания работ и проведение разборов задач для участников и зрителей.
	   \item[–] Жюри Олимпиады: отвечает за проверку работ, формирование результатов и распределение наград. Отдельно назначенные члены жюри осуществляют контроль качества проверки работ проверяющими в регионах.
   \end{enumerate}
	\item Сроки проведения Олимпиады определяются Фондом «Время науки» в календарном плане мероприятий, публикуемом на сайте \surl{timeforscience.ru}. {\bf В 2022 году Олимпиада проводится 12 марта.}
	\item Положение и Регламент Олимпиады публикуются организаторами на сайте Фонда «Время науки» \surl{timeforscience.ru}, а также на сайте Олимпиады \surl{mathnonstop.ru}.
	\item Во время проведения Олимпиады внесение изменений в Положение и Регламент не допускается.
\end{enumerate}

\section{Информационные ресурсы Олимпиады}

\begin{enumerate}
	\item Сайт \surl{mathnonstop.ru} содержит основную информацию об Олимпиаде, необходимую для ее участников, их родителей и учителей. На нем опубликованы итоги олимпиад последних лет, список площадок Олимпиады, время начала и продолжительность Олимпиады, ссылки на разборы заданий, документы, регламентирующие проведение Олимпиады: положение, регламент, согласие на обработку персональных данных, перечень конкурсов, в который включена Олимпиада.
	\item Регистрационная система \surl{rs.mathnonstop.ru} служит для регистрации участников на площадки с одной стороны и сопоставления записей участников с их результатами, работами, наградами, персональными данными~— с другой. Все данные регистрационной системы защищены и хранятся на частном сервере в Санкт-Петербурге. Регистрационная система обеспечивает работу оргкомитета, жюри и взаимодействие между площадками посредством обмена материалами, их централизованного хранения и обработки, а также взаимного контроля проверяющих и руководства жюри.
	\item Страницы \surl{timeforscience.ru} и \surl{vk.com/timeforscience} содержат основную, общую и текущую информацию об организаторе Олимпиады~— Фонде «Время науки» и его проектах.
	\item Для связи с Оргкомитетом Олимпиады по любым вопросам (проблемы с регистрацией, организация площадки, порядок проведения Олимпиады) используется почта
	\begin{center} \surl{mathnonstop@timeforscience.ru}. \end{center}
\end{enumerate}

\section{Участники Олимпиады}

\begin{enumerate}
	\item Олимпиада проводится для учащихся 4–8 классов средних образовательных учреждений Санкт-Петербурга, а также образовательных учреждений других городов России и стран СНГ.
	\item К участию в Олимпиаде допускаются как учащиеся государственных бюджетных образовательных учреждений, так и частных образовательных учреждений, а также обучающиеся, находящиеся на домашнем обучении.
	\item Ограничения по количеству обучающихся, принимающих участие в Олимпиаде, определяются возможностью принять желающих на площадках проведения Олимпиады.
	\item Обучающимся, желающим принять участие в Олимпиаде, следует не позднее установленного срока зарегистрироваться на сайте \surl{rs.mathnonstop.ru}.
	\item Участники, не прошедшие регистрацию, к Олимпиаде не допускаются.
	\item В момент регистрации участник определяет площадку для участия в Олимпиаде. В случае отсутствия мест на данной площадке регистрационная система рекомендует участнику выбрать другую площадку.
	\item Участник обязан принести на Олимпиаду заполненное Согласие на обработку персональных данных. Бланк согласия доступен для скачивания на сайте Олимпиады.
	\item \label{itemBaza}
	   К базовому уровню участия допускаются школьники 4–6 классов
	   образовательных заведений всех типов специализации и
	   школьники 7–8 классов, которые занимаются в общеобразовательных
	   учреждениях в классах, не имеющих математической специализации,
	   и не занимающихся в математических кружках дополнительного образования.
	\item Учащиеся математических школ и лицеев 7–8 классов,
	   учащиеся математических кружков системы дополнительного образования
	   7–8 классов допускаются к участию в Олимпиаде
	   только по профильному уровню.
	\item \label{itemProf}
	   Учащиеся 7–8 классов, которые занимаются в общеобразовательных
	   учреждениях в\linebreak классах, не имеющих математической специализации,
	   и не занимающихся в математических кружках дополнительного образования,
	   также допускаются к профильному варианту по собственному желанию.
	\item При регистрации каждый из участников должен сообщить
	   достоверную информацию о том, какой уровень он заявляет, исходя из
	   требований пп.~\ref{itemBaza}–\ref{itemProf} настоящего Положения.
	   При недостоверной заявке Оргкомитет вправе аннулировать
	   результат участника.
\end{enumerate}

\section{Условия проведения Олимпиады}

\begin{enumerate}
	\item Продолжительность Олимпиады: \begin{center} \begin{tabular}{ll}
		4 класс & 2 часа 30 минут \\
		5 класс & 2 часа 30 минут \\
		6 класс & 2 часа 45 минут \\
		7 класс & 3 часа 00 минут \\
		8 класс & 3 часа 00 минут \\
		Проф. варианты & 3 часа 30 минут \\
	\end{tabular} \end{center}
	\item Условия Олимпиады доступны только на русском языке. Для детей с особыми образовательными потребностями возможно создание специальных версий условий~— например, с увеличенным шрифтом.
	\item Разговоры между участниками не допускаются.
	\item Участникам не разрешается проносить в аудиторию никакие электронные устройства.
	\item Участники Олимпиады, мешающие работе других участников, удаляются из аудиторий.
\end{enumerate}

\section{Публикация результатов}

\begin{enumerate}
	\item Предварительные результаты подводятся Жюри Олимпиады и публикуются на сайте Олимпиады. Одновременно с ними публикуется дата апелляции. 
	\item Апелляция проводится по предварительным результатам Олимпиады.
	\item После апелляции Жюри и Оргкомитет утверждают окончательные результаты и распределяют награды. Окончательные результаты и дата церемонии награждения публикуются на сайте Олимпиады.
\end{enumerate}

\section{Награждение победителей и призеров}

\begin{enumerate}
	\item Победители Олимпиады награждаются дипломами I степени.
	\item Призеры Олимпиады награждаются дипломами II, III степеней и похвальными отзывами 1, 2, 3 степеней.
	\item Итоги Олимпиады публикуются на информационных ресурсах Олимпиады не позднее двух месяцев с момента ее проведения.
	\item Сканы дипломов и похвальных отзывов выкладываются на сайт \surl{mathnonstop.ru} и доступны для скачивания победителями и призерами.
	\item Победителям и призерам Олимпиады вручаются призы и подарки.
	\item Дипломы и призы могут быть получены на церемонии награждения, в Фонде «Время науки» или по почте.
\end{enumerate}

\section{Символика Олимпиады и рекламная деятельность}

\begin{enumerate}
	\item Оргкомитет разрабатывает символику Олимпиады. Символику
	   Олимпиады утверждают Соучредители Олимпиады.
	\item Соучредители и партнеры Олимпиады могут использовать
	   символику Олимпиады вместе с символикой своей организации.
	\item Во всех иных случаях использование символики Олимпиады
	   допускается с письменного разрешения Оргкомитета.
	\item Размещение рекламы организаций, политических партий,
	   объединений во время проведения Олимпиады, в изданиях,
	   брошюрах и иных печатных материалах Олимпиады, допустимы
	   только при разрешении Оргкомитета Олимпиады, при наличии
	   договора с Фондом «Время науки».
\end{enumerate}

\end{document}
